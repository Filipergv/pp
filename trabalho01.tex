\documentclass[11pt]{article}
\usepackage{graphicx}
\usepackage[latin1]{inputenc}   % para os acentos
\usepackage[brazil]{babel}      % para hifeniza\c{c}\~{a}o


\setlength{\oddsidemargin}{-0.5cm}
\setlength{\evensidemargin}{-0.3cm}\setlength{\textwidth}{17.6cm}
\setlength{\textheight}{24cm}\setlength{\topmargin}{-1.0cm}
\setlength{\headheight}{0.0cm} \setlength{\headsep}{0.0cm}

%\twocolumn

\pagestyle{empty}
\usepackage{listings}
\begin{document}

\lstset{basicstyle=\scriptsize,breaklines=true,frame=lines}
\lstloadlanguages{Java}
\lstset{language=Java}
\lstset{numbers=left, numberstyle=\scriptsize, stepnumber=1, numbersep=5pt}

{\small 

\noindent
\rule{1\textwidth}{0.2mm}
\includegraphics[height=6mm, width=11mm]{logo_cin.jpg} \hfill \textbf{{\large Trabalho 01 }} \hfill \includegraphics[height=8mm, width=7mm]{logo_ufpe.jpg} \\
\rule{1\textwidth}{0.2mm}\\

\noindent
{\bf Disciplina:} Programa��o Paralela \hspace{0.5cm} \textbf{Data: 01 de abril de 2014.} \hspace{0.5cm} \hspace{0.5cm}
{\bf Prof.:} Fernando Castor\\[0.5cm]
{\bf 1.} Descreva a arquitetura do seu computador pessoal em termos dos seguintes itens:

\begin{itemize}
\item Frequ�ncia de clock
\item N�mero de n�cleos (f�sicos e virtuais) do processador.
\item N�mero de n�veis de {\em cache} e onde as mem�rias {\em cache} est�o localizadas.
\item Como o processador mant�m coer�ncia das mem�rias {\em cache}
\item As instru��es at�micas n�o-triviais (por exemplo, um {\tt LOAD} ou um {\tt STOR} s�o instru��es ``triviais'' por fazerem apenas uma coisa) que est�o dispon�veis e o que elas fazem.
\item A velocidade da mem�ria que est� rodando na sua m�quina. Ela � r�pida o suficiente para o processador? Sim? N�o? Por qu�?
\end{itemize}

\vspace{0.5cm}
\noindent
{\bf 2.} Voc� j� precisou construir programas paralelos, seja por motivos de estudo, seja por motivos profissionais? Escolha o mais complexo desses programas e descreva-o. Explique porque ele precisa realizar v�rias atividades ao mesmo tempo e em que consistiam essas atividades. Esse programa era ``embara�osamente'' paralelo ou exigia sincroniza��o entre as tarefas? Que problemas voc� enfrentou ao construi-lo (ou ajudar a construi-lo)? 
\end{document} 
